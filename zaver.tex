\section*{Závěr}
Provedli jsme kalibraci "optické sondy anemometru" dvěma metodami. Obě metody daly téměř shodné výsledky, vzdálenost interferenčních plošek v průsečíku je
\begin{equation*}
d_F = \SI{29(1)}{\micro\m} \,.
\end{equation*}

Měření rychlosti částic bylo poněkud neúspěšné (viz diskuze). Histogram na první pohled neodpovídá normálnímu rozdělení. Určili jsme střední rychlost $\mu$ částic a standardní odchylku $\sigma$ rozdělení
\begin{align*}
\mu &= \SI{21.1(11)}{\mm\per\s}\\
\sigma &= \SI{9.3(12)}{\mm\per\s}
\end{align*}