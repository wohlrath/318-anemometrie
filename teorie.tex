\section*{Teoretická část}
Kalibraci "optické sondy anemometru" provedeme dvěma způsoby.
Poprvé změříme přímo $d_F$ vzdálenost dvou interferenčních plošek v průsečíku obou svazků pomocí projekce na stínítko.


Podruhé necháme oba svazky protnout a dopadnout na stínítko. Změříme $d_g$ vzdálenost bodů na stínítku a $l_g$ vzdálenost stínítka od průsečíku svazků. Úhel mezi svazky $\vartheta$ je pak dán
\begin{equation}
\label{eq:kal_geom}
\tan \frac{\vartheta}{2}=\frac{d_g}{2 l_g}
\end{equation}
a
\begin{equation}
\label{eq:kal_inter}
d_F=\frac{\lambda}{2 \sin\frac{\vartheta}{2}} \,,
\end{equation}
kde $\lambda = \SI{632.8}{\nm}$ je vlnová délka použitého laseru.

Aparaturu sestavíme jako v \cite{skripta}. Budeme měřit rychlost částic ve vodě pomocí dopplerovského anemometru. Pokud má částice rychlost $v_x$ ve směru osy x (viz \cite{skripta}, směr v rovině obou svazků, který je kolmý k jejich společnému směru), bude signál tvořen rázy o frekvenci
\begin{equation}
\nu = \frac{v_x}{d_F} = v_x \frac{2\sin\frac{\vartheta}{2}}{\lambda_0} \,.
\end{equation}

Používáme pouze standardní odchylku a chybu nepřímo měřených veličin počítáme metodou přenosu chyb.