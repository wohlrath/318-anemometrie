\documentclass[a4paper]{article}

\usepackage[czech]{babel} %https://github.com/michal-h21/biblatex-iso690
\usepackage[
   backend=biber      % if we want unicode 
  ,style=iso-numeric % or iso-numeric for numeric citation method          
  ,babel=other        % to support multiple languages in bibliography
  ,sortlocale=cs_CZ   % locale of main language, it is for sorting
  ,bibencoding=UTF8   % this is necessary only if bibliography file is in different encoding than main document
]{biblatex}

\usepackage[utf8]{inputenc}
\usepackage{fancyhdr}
\usepackage{amsmath}
\usepackage{amssymb}
\usepackage[left=2cm,right=2cm,top=2.5cm,bottom=2.5cm]{geometry}
\usepackage{graphicx}
\usepackage{pdfpages}
\usepackage{url}

\usepackage{siunitx}
\sisetup{locale = DE  , separate-uncertainty = true} %    kdybych chtel +/-

\usepackage{float}
\newfloat{graph}{htbp}{grp}
\floatname{graph}{Graf}
\newfloat{tabulka}{htbp}{tbl}
\floatname{tabulka}{Tabulka}

\renewcommand{\thefootnote}{\roman{footnote}}

\pagestyle{fancy}
\lhead{Praktikum III - (18) Laserová dopplerovská anemometrie}
\rhead{Vladislav Wohlrath}
\author{Vladislav Wohlrath}

\bibliography{source}

\begin{document}

\begin{titlepage}
\includepdf[pages={1}]{./graficos/titlelist}
\end{titlepage}

\section*{Pracovní úkoly}
\begin{enumerate}
\item Proveďte kalibraci „optické sondy anemometru“. Použijte uspořádání navržené na obr. 4.6 – 5 v \cite{skripta}. Parametry optické sondy získáte jednak měřením vzdálenosti interferenčních plošek v průsečíku laserových paprsků metodou projekce, jednak výpočtem z geometrie uspořádání. Oba výsledky porovnejte.
\item Připravte aparaturu k měření rychlosti částic. Zkontrolujte chod paprsků v detekční optice a vymezte prostorovou dírkovou clonu.
\item Na základě průběhu dopplerovských signálů optimalizujte dopplerovský signál na proudění vody v kyvetě.
\item Změřte frekvence dopplerovského signálu na souboru 60--80 částic. Převeďte hodnoty frekvence na hodnoty rychlostí. Graficky zpracujte rozložení rychlostí ve vodě formou histogramu. Histogram fitujte funkcí normálního rozdělení a z ní stanovte střední hodnotu rychlosti částic a standardní odchylku nalezeného rozdělení.
\item Diskutujte, jaký vliv na výsledek má to, že parametry optické sondy jsou měřeny ve vzduchu, zatímco měření rychlostí částic probíhá ve vodě.
\end{enumerate}

%Teoretická část
\section*{Teoretická část}
Kalibraci "optické sondy anemometru" provedeme dvěma způsoby.
Poprvé změříme přímo $d_F$ vzdálenost dvou interferenčních plošek v průsečíku obou svazků pomocí projekce na stínítko.


Podruhé necháme oba svazky protnout a dopadnout na stínítko. Změříme $d_g$ vzdálenost bodů na stínítku a $l_g$ vzdálenost stínítka od průsečíku svazků. Úhel mezi svazky $\vartheta$ je pak dán
\begin{equation}
\label{eq:kal_geom}
\tan \frac{\vartheta}{2}=\frac{d_g}{2 l_g}
\end{equation}
a
\begin{equation}
\label{eq:kal_inter}
d_F=\frac{\lambda}{2 \sin\frac{\vartheta}{2}} \,,
\end{equation}
kde $\lambda = \SI{632.8}{\nm}$ je vlnová délka použitého laseru.

Aparaturu sestavíme jako v \cite{skripta}. Budeme měřit rychlost částic ve vodě pomocí dopplerovského anemometru. Pokud má částice rychlost $v_x$ ve směru osy x (viz \cite{skripta}, směr v rovině obou svazků, který je kolmý k jejich společnému směru), bude signál tvořen rázy o frekvenci
\begin{equation}
\nu = \frac{v_x}{d_F} = v_x \frac{2\sin\frac{\vartheta}{2}}{\lambda_0} \,.
\end{equation}

Používáme pouze standardní odchylku a chybu nepřímo měřených veličin počítáme metodou přenosu chyb.

%Výsledky měření
\section*{Výsledky měření}

Provedli jsme kalibraci "optické sondy anemometru" oběma metodami.
Změřili jsme vzdálenost mezi 10 interferenčními proužky (10 mezer) \SI{0.28(2)}{\mm}, chybu odhadujeme s přihlédnutím k přesnosti mikrometru a metody. Změřili jsme $d_g = \SI{2.50(5)}{\cm}$ a $l_g = \SI{116(1)}{\cm}$. Dostali jsme vzdálenost interferenčních ploch


\begin{align*}
\begin{split}
\text{měřením interferenčních proužků} & \qquad d_{F_i} = \SI{28(2)}{\micro\m} \\
\text{z geometrie uspořádání} & \qquad d_{F_g} = \SI{29.4(10)}{\micro\m}
\end{split}
\end{align*}

Protože se oba výsledky dobře shodují, budeme dále používat jejich průměr vážený chybou $d_F = \SI{29(1)}{\micro\m}$.

Změřili jsme celkem 61 částic, naměřené hodnoty jsou v tabulce v příloze. Histogram rychlostí částic je v grafu \ref{g:hist}.
Histogram jsme v programu GNUPLOT fitovali funkcí normálního rozdělení
\begin{equation*}
f(v_x)=A \cdot \exp{\left(-\frac{(v_x-\mu)^2}{2\sigma^2}\right)} \,.
\end{equation*}
Výsledné parametry jsou
\begin{align*}
\mu &= \SI{21.1(11)}{\mm\per\s}\\
\sigma &= \SI{9.3(12)}{\mm\per\s}
\end{align*}

\begin{graph}[htbp] 
\centering
% GNUPLOT: LaTeX picture with Postscript
\begingroup
  \makeatletter
  \providecommand\color[2][]{%
    \GenericError{(gnuplot) \space\space\space\@spaces}{%
      Package color not loaded in conjunction with
      terminal option `colourtext'%
    }{See the gnuplot documentation for explanation.%
    }{Either use 'blacktext' in gnuplot or load the package
      color.sty in LaTeX.}%
    \renewcommand\color[2][]{}%
  }%
  \providecommand\includegraphics[2][]{%
    \GenericError{(gnuplot) \space\space\space\@spaces}{%
      Package graphicx or graphics not loaded%
    }{See the gnuplot documentation for explanation.%
    }{The gnuplot epslatex terminal needs graphicx.sty or graphics.sty.}%
    \renewcommand\includegraphics[2][]{}%
  }%
  \providecommand\rotatebox[2]{#2}%
  \@ifundefined{ifGPcolor}{%
    \newif\ifGPcolor
    \GPcolorfalse
  }{}%
  \@ifundefined{ifGPblacktext}{%
    \newif\ifGPblacktext
    \GPblacktexttrue
  }{}%
  % define a \g@addto@macro without @ in the name:
  \let\gplgaddtomacro\g@addto@macro
  % define empty templates for all commands taking text:
  \gdef\gplbacktext{}%
  \gdef\gplfronttext{}%
  \makeatother
  \ifGPblacktext
    % no textcolor at all
    \def\colorrgb#1{}%
    \def\colorgray#1{}%
  \else
    % gray or color?
    \ifGPcolor
      \def\colorrgb#1{\color[rgb]{#1}}%
      \def\colorgray#1{\color[gray]{#1}}%
      \expandafter\def\csname LTw\endcsname{\color{white}}%
      \expandafter\def\csname LTb\endcsname{\color{black}}%
      \expandafter\def\csname LTa\endcsname{\color{black}}%
      \expandafter\def\csname LT0\endcsname{\color[rgb]{1,0,0}}%
      \expandafter\def\csname LT1\endcsname{\color[rgb]{0,1,0}}%
      \expandafter\def\csname LT2\endcsname{\color[rgb]{0,0,1}}%
      \expandafter\def\csname LT3\endcsname{\color[rgb]{1,0,1}}%
      \expandafter\def\csname LT4\endcsname{\color[rgb]{0,1,1}}%
      \expandafter\def\csname LT5\endcsname{\color[rgb]{1,1,0}}%
      \expandafter\def\csname LT6\endcsname{\color[rgb]{0,0,0}}%
      \expandafter\def\csname LT7\endcsname{\color[rgb]{1,0.3,0}}%
      \expandafter\def\csname LT8\endcsname{\color[rgb]{0.5,0.5,0.5}}%
    \else
      % gray
      \def\colorrgb#1{\color{black}}%
      \def\colorgray#1{\color[gray]{#1}}%
      \expandafter\def\csname LTw\endcsname{\color{white}}%
      \expandafter\def\csname LTb\endcsname{\color{black}}%
      \expandafter\def\csname LTa\endcsname{\color{black}}%
      \expandafter\def\csname LT0\endcsname{\color{black}}%
      \expandafter\def\csname LT1\endcsname{\color{black}}%
      \expandafter\def\csname LT2\endcsname{\color{black}}%
      \expandafter\def\csname LT3\endcsname{\color{black}}%
      \expandafter\def\csname LT4\endcsname{\color{black}}%
      \expandafter\def\csname LT5\endcsname{\color{black}}%
      \expandafter\def\csname LT6\endcsname{\color{black}}%
      \expandafter\def\csname LT7\endcsname{\color{black}}%
      \expandafter\def\csname LT8\endcsname{\color{black}}%
    \fi
  \fi
  \setlength{\unitlength}{0.0500bp}%
  \begin{picture}(10204.00,5668.00)%
    \gplgaddtomacro\gplbacktext{%
      \csname LTb\endcsname%
      \put(814,704){\makebox(0,0)[r]{\strut{} 0}}%
      \csname LTb\endcsname%
      \put(814,1331){\makebox(0,0)[r]{\strut{} 2}}%
      \csname LTb\endcsname%
      \put(814,1957){\makebox(0,0)[r]{\strut{} 4}}%
      \csname LTb\endcsname%
      \put(814,2584){\makebox(0,0)[r]{\strut{} 6}}%
      \csname LTb\endcsname%
      \put(814,3210){\makebox(0,0)[r]{\strut{} 8}}%
      \csname LTb\endcsname%
      \put(814,3837){\makebox(0,0)[r]{\strut{} 10}}%
      \csname LTb\endcsname%
      \put(814,4463){\makebox(0,0)[r]{\strut{} 12}}%
      \csname LTb\endcsname%
      \put(814,5090){\makebox(0,0)[r]{\strut{} 14}}%
      \csname LTb\endcsname%
      \put(946,484){\makebox(0,0){\strut{} 0}}%
      \csname LTb\endcsname%
      \put(1832,484){\makebox(0,0){\strut{} 5}}%
      \csname LTb\endcsname%
      \put(2718,484){\makebox(0,0){\strut{} 10}}%
      \csname LTb\endcsname%
      \put(3604,484){\makebox(0,0){\strut{} 15}}%
      \csname LTb\endcsname%
      \put(4490,484){\makebox(0,0){\strut{} 20}}%
      \csname LTb\endcsname%
      \put(5377,484){\makebox(0,0){\strut{} 25}}%
      \csname LTb\endcsname%
      \put(6263,484){\makebox(0,0){\strut{} 30}}%
      \csname LTb\endcsname%
      \put(7149,484){\makebox(0,0){\strut{} 35}}%
      \csname LTb\endcsname%
      \put(8035,484){\makebox(0,0){\strut{} 40}}%
      \csname LTb\endcsname%
      \put(8921,484){\makebox(0,0){\strut{} 45}}%
      \csname LTb\endcsname%
      \put(9807,484){\makebox(0,0){\strut{} 50}}%
      \put(176,3053){\rotatebox{-270}{\makebox(0,0){\strut{}počet částic}}}%
      \put(5376,154){\makebox(0,0){\strut{}$v_x$ (\si{\mm\per\s})}}%
      \put(4869,1644){\makebox(0,0)[l]{\strut{}$\mu$}}%
      \put(3782,3523){\makebox(0,0)[l]{\strut{}$2\sigma$}}%
    }%
    \gplgaddtomacro\gplfronttext{%
      \csname LTb\endcsname%
      \put(8820,5230){\makebox(0,0)[r]{\strut{}histogram}}%
      \csname LTb\endcsname%
      \put(8820,5010){\makebox(0,0)[r]{\strut{}normální rozdělení}}%
    }%
    \gplbacktext
    \put(0,0){\includegraphics{hist}}%
    \gplfronttext
  \end{picture}%
\endgroup

\caption{Histogram rychlostí částic}
\label{g:hist}
\end{graph}

\begin{graph}[htbp] 
\centering
\input{typ.tex}
\caption{Pozorovaný diferenciální dopplerovský signál}
\label{g:typ}
\end{graph}

%Diskuze výsledků
\section*{Diskuze}
Obě metody kalibrace "optické sondy anemometru" daly přibližně stejný výsledek. Geometrickou metodu považujeme za poněkud přesnější. Při projekční metodě vznikla chyba především při určování, zda je proužek již zakryt, nebo ne.


Běhěm samotného měření se nepodařilo odizolovat periodický šum, který byl mnohokrát silnější než měřený signál. Pravděpodobně šlo o \SI{50}{\Hz} šum ze sítě. Tento šum do značné míry znemožnil měření. Byli jsme nuceni měřit i částice, které bychom za normálních okolností neměřili.


Histogram na grafu \ref{g:hist} příliš nepřipomíná normální rozdělení. Vinu připisujeme zmíněnému šumu a jím způsobeným problémům. 


Voda má index lomu rozdílný od vzduchu, takže při přechodu se změní vlnová délka $\lambda$ i úhel $\vartheta$. Ve vzorci \eqref{eq:kal_inter} ale tyto dvě změny působí proti sobě
\begin{equation*}
d_F=\frac{\lambda\prime}{2} \frac{1}{\sin\frac{\vartheta\prime}{2}}=\frac{\lambda}{2n} \frac{n}{sin\frac{\vartheta}{2}} \,,
\end{equation*}
kde $n$ je index lomu vody a čárkované veličiny značí hodnoty ve vodě. Skutečnost, že jsme "optickou sondu anemometru" kalibrovali ve vzduchu, tedy na výsledek nemá vliv.

%Závěr
\section*{Závěr}
Provedli jsme kalibraci "optické sondy anemometru" dvěma metodami. Obě metody daly téměř shodné výsledky, vzdálenost interferenčních plošek v průsečíku je
\begin{equation*}
d_F = \SI{29(1)}{\micro\m} \,.
\end{equation*}

Měření rychlosti částic bylo poněkud neúspěšné (viz diskuze). Histogram na první pohled neodpovídá normálnímu rozdělení. Určili jsme střední rychlost $\mu$ částic a standardní odchylku $\sigma$ rozdělení
\begin{align*}
\mu &= \SI{21.1(11)}{\mm\per\s}\\
\sigma &= \SI{9.3(12)}{\mm\per\s}
\end{align*}


\printbibliography[title={Seznam použité literatury}]

\end{document}